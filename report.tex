%%%%%%%%%%%%%%%%%%%%%%%%%%%%%%%%%%%%%%%%%
% Large Colored Title Article
% LaTeX Template
% Version 1.1 (25/11/12)
%
% This template has been downloaded from:
% http://www.LaTeXTemplates.com
%
% Original author:
% Frits Wenneker (http://www.howtotex.com)
%
% Modified by:
% Julian Kirsch
%
% License:
% CC BY-NC-SA 3.0 (http://creativecommons.org/licenses/by-nc-sa/3.0/)
%
%%%%%%%%%%%%%%%%%%%%%%%%%%%%%%%%%%%%%%%%%

%----------------------------------------------------------------------------------------
%	PACKAGES AND OTHER DOCUMENT CONFIGURATIONS
%----------------------------------------------------------------------------------------

\documentclass[DIV=calc, paper=a4, fontsize=11pt, twocolumn]{scrartcl}

\usepackage{lipsum}
\usepackage[english]{babel}
\usepackage{microtype}
\usepackage[T1]{fontenc}
\usepackage{amsmath,amsfonts,amsthm}
\usepackage{booktabs}
\usepackage{sectsty}
\usepackage{url}
\usepackage{listings}
\usepackage[usenames,dvipsnames]{xcolor}
\usepackage[font=small,format=plain,labelfont=bf,up,textfont=it,up]{caption}

\usepackage{fancyhdr}
\usepackage{lastpage}

% ------------- neu:
\usepackage{pgf,tikz}
\usepackage{xspace}

\graphicspath{{images/}}

% -------------- custom commands: ---------------
\newcommand{\sse}{S\textsuperscript{2}E\xspace}
\newcommand{\app}{\textit{SuperTaxCalcPro}\xspace}

%todo anfang
  \newcommand{\todo}[1]{
  % Add to todo list
  \addcontentsline{tdo}{todo}{\protect{#1}}
  %
  \begin{tikzpicture}[remember picture, baseline=-0.35ex]
      \node [coordinate] (inText) {};
  \end{tikzpicture}
  %
  % Make the margin par
  \marginpar{
      \begin{tikzpicture}[remember picture]
          \definecolor{orange}{rgb}{1,0.5,0}
   
          \draw node[draw=black, fill=orange, text width = 0.2cm, text height = 0.1cm] (inNote)
                   {#1};
      \end{tikzpicture}
  }
  %
  \begin{tikzpicture}[remember picture, overlay]
      \draw[draw = orange, thick]
          ([yshift=-0.2cm] inText)
              -| ([xshift=-0.2cm] inNote.west)
              -| (inNote.west);
  \end{tikzpicture}
  %
  }
% todo ende


\sloppy
\hbadness 10000
\renewcommand*\rmdefault{ppl}\normalfont\upshape
\renewcommand{\topfraction}{0.85}
\renewcommand{\textfraction}{0.1}
\renewcommand{\floatpagefraction}{0.75}
\renewcommand{\UrlFont}{\small\tt}
\allsectionsfont{\color{tumblue}\usefont{OT1}{phv}{b}{n}}

\usepackage[osf, sc]{mathpazo}
\usepackage{helvet}

\definecolor{tumblue}{RGB}{0,101,189}

% https://en.wikibooks.org/wiki/LaTeX/Source_Code_Listings
\lstset{
  numbers=left,
  numberstyle=\tiny\color{gray},
  stepnumber=1,
  numbersep=5pt,
  showspaces=false,
  showstringspaces=false,
  showtabs=false,
  frame=single,
  rulecolor=\color{black},
  tabsize=8,
  captionpos=b,
  breaklines=true,
  breakatwhitespace=false,
  language=C,
  keywordstyle=\bfseries\color{OliveGreen},
  commentstyle=\itshape\color{Mahogany},
  stringstyle=\color{BrickRed},
  keywordstyle=[2]{\color{Cyan}},
  escapechar=ß,
  xleftmargin=8pt,
  xrightmargin=3pt,
  basicstyle=\scriptsize,
  morekeywords={u32, __u32, __be32, __le32,
  		u16, __u16, __be16, __le16,
	        u8,  __u8,  __be8,  __le8,
	        size_t, ssize_t}
}

% https://tex.stackexchange.com/questions/51645/
%  x86-64-assembler-language-dialect-for-the-listings-package
\lstdefinelanguage
   [x86_64]{Assembler}
   [x86masm]{Assembler}
   % with these extra keywords:
   {morekeywords={CDQE, CQO, CMPSQ, CMPXCHG16B, JRCXZ, LODSQ, MOVSXD,
                  POPFQ, PUSHFQ, SCASQ, STOSQ, IRETQ, RDTSCP, SWAPGS,
                  rax, rdx, rcx, rbx, rsi, rdi, rsp, rbp,
                  r8, r8d, r8w, r8b, r9, r9d, r9w, r9b}}

\usepackage{lettrine}
\newcommand{\initial}[1]{
\lettrine[lines=3,lhang=0.3,nindent=0em]{
\color{tumblue}
{\textsf{#1}}}{}}



%----------------------------------------------------------------------------------------
%	TITLE SECTION
%----------------------------------------------------------------------------------------

\usepackage{titling}
\newcommand{\HorRule}{\color{tumblue} \rule{\linewidth}{1pt}}

\pretitle{\thispagestyle{noheadings}\vspace{-30pt}
  \begin{flushleft} \HorRule \fontsize{30}{40} \usefont{OT1}{phv}{b}{n} \color{tumblue} \selectfont}

\title{Selective Symbolic Execution}
% \\ \large Analysis of user space binaries using the \sse platform}

\posttitle{\par\end{flushleft}\vskip 0.5em}

\preauthor{\begin{flushleft}\large \lineskip 0.5em \usefont{OT1}{phv}{b}{sl}
  \color{tumblue}}

% Please leave this as it is for the 1st draft as our "peer-review"
% is supposed to take place anonymously (like in real life)
\author{Anonymous}

\postauthor{\footnotesize \usefont{OT1}{phv}{m}{sl} \color{Black} % Configuration for the institution name
, Technische Universit\"at M\"unchen

\par\end{flushleft}\HorRule}
\date{}

%----------------------------------------------------------------------------------------
\makeatletter
\let\docauthor\@author
\makeatother
\makeatletter
\let\doctitle\@title
\makeatother


\fancypagestyle{headings}{
  \lhead{}
  \chead{}
  \rhead{\usefont{OT1}{phv}{m}{sc}\footnotesize \doctitle }

  % Footers
  \lfoot{\usefont{OT1}{phv}{m}{sc}\footnotesize \docauthor }
  \cfoot{}
  \rfoot{\usefont{OT1}{phv}{m}{sc}\footnotesize Page \thepage\ of \pageref{LastPage}}

  \renewcommand{\headrulewidth}{0.4pt}
  \renewcommand{\footrulewidth}{0.4pt}
}

\fancypagestyle{plain}{
  \fancyhf{}

  % Footers
  \lfoot{\usefont{OT1}{phv}{m}{sc}\footnotesize \docauthor }
  \cfoot{}
  \rfoot{\usefont{OT1}{phv}{m}{sc}\footnotesize Page \thepage\ of \pageref{LastPage}}

  \renewcommand{\headrulewidth}{0.0pt}
  \renewcommand{\footrulewidth}{0.4pt}
}
\lfoot{\usefont{OT1}{phv}{m}{sc}\footnotesize \docauthor }
\cfoot{}
\rfoot{\usefont{OT1}{phv}{m}{sc}\footnotesize Page \thepage\ of \pageref{LastPage}}


\begin{document}

% Stop whining, \maketitle
\newcommand{\undefinedpagestyle}{}
\maketitle
\pagestyle{headings}

%----------------------------------------------------------------------------------------
%	ABSTRACT
%----------------------------------------------------------------------------------------

% The first character should be within \initial{}
\initial{T}\textbf{his paper describes the exemplary application of selective symbolic execution techniques for the analysis of a binary file in user mode.
Goal of this study is to search the binary for possible privacy issues like unwanted leakage of personal data.
Investigation will be done using \sse, a powerful platform for selective symbolic execution of large software systems.
}

%----------------------------------------------------------------------------------------
%	ARTICLE CONTENTS
%----------------------------------------------------------------------------------------

\bigskip

\section{Introduction}

% use cases (sel symb exe paper kap 2) hier hinein?

Frequently developers need to understand software systems.
In a very simple case they just analyse their own code or test the interaction of own programs with other components or with the surrounding environment in general.
Testing self-written programs conceptually permits the application of the whole arsenal of analysis techniques.


Things become interesting when analysis has to be performed without access to source code or documentation.
Scenarios for this situation include the need to check proprietary third party software for interoperability, performance, unwanted side effects, and much more.
% on existing servers
Security-critical environments additionally require reliable guarantees of the benignity of all employed software.


One mighty solution for such system analysis is the \sse platform developed at the Swiss Federal Institute of Technology in Lausanne (EPFL) \cite{chip11s2e}.
Its goal is to provide a tool set for rapid development of analysis tools like performance profilers, bug finders, reverse engineering solutions and the like \cite{chip12s2e}.
\sse combines several key characteristics:

\medskip
1.) The ability to explore entire \textit{families of execution paths} helps to obtain reliable information about the target system.
Abstracting from single-path exploration to sets of execution paths which share specific properties is vital for predictive analyses.
This technique can for example prove the non-existence of critical corner cases which might be overlooked by other testing strategies.
% ...and hence provides more reliable results.
%Particularly for predictive tasks the analysis of sets of execution paths which share specific properies
%This is often necessary for predictive analyses, where classical single-path exploration techniques fail to provide reliable results.

\medskip
2.) \textit{In-vivo analysis}, meaning the analysis of a program within its real-world environment (libraries, kernel, drivers, etc.), facilitates extremely realistic and accurate results.

\medskip
3.) Working directly on \textit{binaries} further increases the degree of realism in system analyses, as it allows to include closed source modules into the investigation.

\bigskip

As an exemplary showcase for the power of the \sse platform and its underlying concepts this paper will perform a thorough analysis of a binary file in user mode.
The scenario assumes that this program was found somewhere in the internet and claims to be a useful freeware tool.
\sse shall help to investigate whether the binary compromises the user's privacy, for instance by leaking private data to the internet.


%What I do is bla...
%...justify the choice of \sse as platform for system analysis.\todo{?}

\bigskip

Chapter \ref{sec:s2e} explains the theoretical concepts of selective symbolic execution.
Chapter \ref{sec:platform} then introduces \sse, the platform which builds upon all techniques described before.
Coming to the practical part, chapter \ref{sec:proj} lays out the concrete analysis scenario and formulates research questions.
Chapter \ref{sec:impl} describes how \sse can be applied in this scenario, followed by an explanation of \sse results in chapter \ref{sec:ana}.
Possible further research related to this topic is mentioned in chapter \ref{sec:outlook}, together with some selected related work in chapter \ref{sec:rel_work}.
Finally, chapter \ref{sec:conclusion} summarises and concludes this paper.

%\newpage
\iffalse
§1	Introduction (Motivation)
		> Motivation, warum es einen Bedarf für etwas wie S2E gibt
		> Kurze Vorschau, was man damit Tolles machen kann
			-> Pfadanalyse -> Prediction of system behaviour
			-> Analyse in natürlicher Umgebung -> “in-vivo”

§2	Selective Symbolic Execution
		> Theorie-Teil
		> Was ist Symbolic Execution?
		> Was kann Selective Symbolic Execution besser?
			(Concrete -> symbolic transition usw.)
		> Konsistenzmodelle (wird hier evtl. schwierig, das richtige Maß 
			zu finden, um die Sache auf wenig Platz zu verstehen)

§3	The S2E Platform
		> Architektur
		> Funktionsweise
		> Selektoren + Analysatoren

§4	Project idea: explore privacy issues in a sample binary
		> Plan darlegen: Programme könnten unerwünscht Infos preisgeben.
		> Daher: Eigenes kleines Programm, das … macht.

§5 	Implementation (of the test case using S2E)
		> Vorgehen
		> Verwendete Konsistenzmodelle
		> Arbeitsweise von Selektoren/Analysatoren

§6	Interpretation of S2E analysis output
		> Execution Traces
		> Gefundene Privacy-Probleme
		> Eventuell nicht gefundene Sachen
		> Probleme bei der Analyse

§7	Outlook
		> Was man sonst noch Tolles machen könnte.
		> Anwendung auf echte Malware, etc…

		> Kommt das hier zu früh? Evtl. auch in den Schlussteil...

§8	Related Work
		> Was haben andere mit S2E in ähnlicher Richtung gemacht?
		> Z.B. "Finding Trojan Message Vulnerabilities in Distributed Systems”

§9	Conclusion
		> Zusammenfassung: was habe ich gemacht?
		> Fazit: war die Sache sinnvoll und erfolgreich?
\fi







\section{Selective Symbolic Execution}\label{sec:s2e}

% - evtl. einbauen:
% ------ KLEE arbeitet auf Code -> S2E auf binaries!


% -------- symbolic execution -------- %
\textbf{Symbolic execution} is an advanced analysis technique particularly suited for automated software testing \todo{ct} and malware analysis \todo{ct}.
Instead of concrete input (7, ``string'', ...) symbolic execution uses symbolic values ($\lambda$, $\beta$, ...) when processing code.
Assignments in the program path have impacts on these symbolic values.
The integer calculation $x = x - 2$, for instance, would update the symbolic expression representing the input $x$ to $\lambda - 2$.
Conditional statements (if <condition> then ... else ...) fork program execution into two new paths.
Both paths are then constrained by an additional condition, the `then' branch with the if-condition and the `else' branch with the negated if-condition respectively.
\todo{Bild > Text}

\begin{figure}
\includegraphics[width=\columnwidth]{symb_exe_2}
\caption{Execution tree with path constraints for the symbolic variable $rpm$ \cite{chip14s2e}}
\label{fig:arch}
\end{figure}

Following this procedure results in a tree-like structure of constrained symbolic expressions.
A constraint solver can now take all constraints along one execution path as input and find one concrete input (e.g., $\lambda = 5$) which would lead to the program following exactly this path.
Such results greatly alleviate writing reproducible test cases \cite{chip09sel}.

On a technical level, symbolic execution engines save state information (program memory, constraint information, ...) in a custom data structure.
Each conditional statement involving symbolic values results in a $fork$ of the program state.
The two newly created branches are completely independent and can therefore be processed in parallel.

% -------path explosion
But the exponential growth of conditionals soon reveals scaling problems of this forking strategy.
Despite heavy research on optimisations mitigating this \textit{path explosion} problem \todo{ct} only relatively small programs ($\cong$ thousands of lines of code) can be analysed symbolically \cite{chip09sel}.

% -------interaction with env
%In addition to the path explosion problem, classical symbolic execution ... interaction with the environment...
Additionally, symbolic execution faces problems when the program under analysis \textit{interacts with its environment}.
If it calls a system library like $libc$, in theory the whole system stack including invoked libraries, operating system and drivers would have to be executed symbolically.
Considering the path explosion problem mentioned before, the resulting complexity makes such a profound analysis hardly feasible.

% --------standard sol
One way to solve this problem is to build abstract models of the program's environment \todo{ct}.
However, due to the complexity of real-world systems, building a model of the entire system is both tedious and unnecessary - the user usually wants to analyse one single program and not the whole system \cite{chip09sel}.

\bigskip

% -------- selective symbolic execution -------- %
In order to overcome typical problems of conventional symbolic execution, Chipounov et al.~from the Swiss Federal Institute of Technology in Lausanne (EPFL) developed the concept of \textbf{selective symbolic execution} (\sse) \cite{chip09sel}.
Based on a virtual execution platform \sse gives users the illusion of running the entire system symbolically.
By limiting the scope of interest (i.e.~which parts of the system should be executed symbolically), users can effectively restrain the path explosion problem.
Program code within this defined scope is executed symbolically, whereas out-of-scope parts, which are irrelevant to the analysis, switch to concrete execution.
%One of the main contributions of the EPFL team around Chipounov is the transparent and consistent management of switching between symbolic and concrete execution modes.

% puning for scalability -> bringen?

Definition of the scope of interest (what to execute symbolically) is highly flexible.
Users may specify whole executables, code regions, or even single variables to be executed symbolically. Everything else will be treated concretely.

%This forth and back conversion -> challenge!
But since on a technical level symbolical and concrete execution are handled very differently - concrete code may run natively while symbolic instructions need to be emulated - switching back and forth these two modes is a major challenge.
Hence one of the main contributions of the EPFL team around Chipounov is the transparent and consistent management of switching between symbolic and concrete execution modes.

\todo{kill}
\cite{chip09sel}
\cite{chip11s2e}
\cite{chip12s2e}
\cite{chip14s2e}





% -------- Konsistenzmodelle -------- %

bla bla bla



%%%%%%%%%%%%%%%%%%%%%%%%%%%%%%%%%%%%%%%%%%%%%%%%%%%%%%%
\iffalse
§2	Selective Symbolic Execution
		> Theorie-Teil
		> Was ist Symbolic Execution?
		> Was kann Selective Symbolic Execution besser?
			(Concrete -> symbolic transition usw.)
		> Konsistenzmodelle (wird hier evtl. schwierig, das richtige Maß 
			zu finden, um die Sache auf wenig Platz zu verstehen)
\fi

\section{The \sse Platform}\label{sec:platform}

Based on the concepts described in the previous chapter, Chipounov and his team implemented the \sse platform, an open source framework for writing custom system analysis tools.
\sse employs the theoretical concepts of selective symbolic execution by running the system under analysis in a virtual machine and treating code within the scope of interest as symbolic.
These symbolic parts are translated into an intermediate representation (\todo{.}), while irrelevant instructions are directly passed to the host for native execution.


Technical backbone of \sse are the virtual machine hypervisor QEMU \cite{qemu, qemu05}, the symbolic execution engine KLEE \cite{klee, klee08} and the LLVM compiler infrastructure \cite{llvm, llvm04}.
Figure \ref{fig:arch} gives an overview of how these technologies are integrated into the \sse platform.
The top of the picture depicts the software stack of the guest system (=the system under analysis), which is managed by QEMU.
\sse is not restricted to user land applications, but also allows inspection on deeper levels (e.g., operating system functions).


\begin{figure}
\includegraphics[width=\columnwidth]{s2e_arch}
\caption{Architecture of the \sse platform \cite{chip12s2e}}
\label{fig:arch}
\end{figure}


For easier emulation, QEMU translates machine code of the guest system into an intermediate representation, called $microoperations$.
\sse's dynamic binary translator (DBT) splits the resulting microoperations into those that need to be explored symbolically and those which may run concretely.
All concrete microoperations are directly converted into host instructions.
Symbolic expressions, on the other hand, are prepared for being executed on the KLEE engine.
This requires microoperations to be translated into the LLVM intermediate representation, called LLVM Bitcode in figure \ref{fig:arch}.

\sse's execution engine, which is an extension to QEMU's execution engine, now manages the operation of the platform.
In an endless loop it asks the DBT for new guest code.
Depending on the result, instructions can either be run straight on the host system or are fed into the KLEE symbolic execution engine.
% --- - --- - Emulation Helpers bringen?

In order to keep the mix of symbolic and concrete execution consistent, \sse stores state (VM CPU, memory, ...) centrally, by consolidating QEMU and KLEE data structures and managing them in a single machine state representation.
% - --- - - - Effizienz bringen?


Users work with \sse by writing selection and analysis plugins or by simply configuring \sse's standard plugins according to their needs.
Plugins subscribe to system-wide events (e.g., $onInstrExecution$) and can perform logging/monitoring tasks or even manipulate the system state.

Configuration usually starts with defining what parts of the system to explore symbolically.
This can for example be done with \sse's selection plugin $CodeSelector$, which restricts symbolic execution to a specified module or code region.

Standard analysis plugins allow users to find bugs ($WinBugCheck$), monitor memory ($MemoryChecker$), study performance characteristics ($PerformanceProfiler$) and much more (see \cite{chip14s2e}, p.~50).
% - ------- hier mehr?
\todo{m?}



% ---------------------------------------------------------------- %
\iffalse
§3	The S2E Platform
		> Architektur
		> Funktionsweise
		> Selektoren + Analysatoren
\fi
\section{Project Idea}\label{sec:proj}

The practical part of this project strives to explore privacy issues in a sample binary.

In order to make life easier, many people use little freeware applications on a regular basis.
But most of these programs are proprietary and have to be trusted without any knowledge of their functioning.
Real malware (Trojan horses, spyware, ...) is usually detected rather quickly by anti-virus software and can often be blocked effectively.
However, between unambiguous malware and thoroughly benign software many shades of grey can be found.

This work will focus on the scenario that an application (intentionally or unintentionally) leaks delicate private data without the user's consent or knowledge.


Due to the difficulty to find a real-world program which shows exactly this desired behaviour and also in general the complexity of real-world applications, the showcase described here bases on a little self-written program.

The software works as follows:

What we do not want is...


All analysis will be done using the \sse platform.
\todo{m?}

\iffalse
§4	Project idea: explore privacy issues in a sample binary
		> Plan darlegen: Programme könnten unerwünscht Infos preisgeben.
		> Daher: Eigenes kleines Programm, das … macht.
\fi
\section{Implementation}\label{sec:impl}



\iffalse
§5 	Implementation (of the test case using S2E)
		> Vorgehen
		> Verwendete Konsistenzmodelle
		> Arbeitsweise von Selektoren/Analysatoren
\fi
\section{Analysis of \sse Output}\label{sec:ana}

Each execution of \sse creates a folder for analysis output.
Apart from the general log files, where all plugins write important messages, some plugins also create separate files.
One file for memory ..., one with LLVM bytecode, one ...\todo{..}

The most important information about \sse's general execution is accumulated in \textit{messages.txt}.
Backbone of this file is information about all execution paths, at which memory addresses they were forked, how they are constrained, when and why they were terminated, and much more.
%But it is also used by other plugins...

Thanks to the clear structure of \textit{messages.txt}, it can be used as input for a custom Python script. This script was written in order to visualise the tree of execution paths graphically, which serves as a great help with understanding path forking behaviour.
Figure \ref{} shows the root\todo{?} of the tree of execution paths of \app.
Each box represents an execution state, lines pointing to another state symbolise a fork of the execution state.
The hex number at the origin of each edge is the memory address in which the state was forked.
Edges are labelled with all constraints that need to be respected in the corresponding execution state.
Blue state boxes indicate one connection to the internet so far, yellow two and red three.
The text below each leaf shows the output of the \textit{TestCaseGenerator} plugin for this execution path.

It is only the root because....

The whole graph (without constraints) looks as folllows:...

\begin{figure}
\includegraphics[width=\columnwidth]{small}
\caption{Graph}
\label{fig:tree}
\end{figure}



\iffalse
§6	Interpretation of S2E analysis output
		> Execution Traces
		> Gefundene Privacy-Probleme
		> Eventuell nicht gefundene Sachen
		> Probleme bei der Analyse
\fi
\section{Outlook}\label{sec:outlook}

This chapter delineates a few suggestions for further interesting research topics related to the project of this paper, starting with concrete ideas for further analysis of the scenario defined in chapter \ref{sec:proj}.

\medskip
Due to the path explosion problems described above it would obviously be interesting to run \sse on a more powerful machine.
This would allow a more comprehensive observation strategy, like using more symbolic variables.

\medskip
Very fine-grained scripting of \sse's configuration file or even interfering in \sse's C++ source code could handle state forking and symbolic-concrete switching more intelligently.
Out of the box, \sse of course does not know our analysis goals and pursues every state.
Adapting this internal behaviour to personal needs should reduce execution time and effort significantly.

\medskip
During the analysis done for this paper, the server responding to \app's messages was available as binary and always running on the guest VM.
This is certainly a very unlikely situation when analysing real-world programs.
The server might not be available or respond with much latency.
It could potentially block the overload of requests triggered by our test execution, or we might not want to reveal our analysis efforts.

In all of these cases symbolic execution must become independent of the server's presence.
Besides, \app's execution tree currently does not depend on server responses.
If the application contained switches like `\textit{if srv\_response == x then do ...}', then the consistency model applied in chapters \ref{sec:impl} and \ref{sec:ana} would fail.

Such a scenario would require \sse execution to be configured for using a \textit{overapproximate consistency (RC-OC)} or \textit{control flow graph consistency (RC-CC)} model (see \cite{chip14s2e}, p.~26ff).
In the \app example, this would mean that responses from the server have to be intercepted and replaced with unconstrained symbolic variables.
Practically speaking, the whole function \textit{send\_data} would have to be replaced with a symbolic variable.
As a consequence, \sse would of course produce lots of states based on completely absurd potential server responses.
We simply do not know how the server responds to a certain client message nor how a valid server response looks like.
But this is the only way to make sure that \sse eventually also finds all valid server responses in the absence of any knowledge of how the server works. 


\medskip
Moving away from the the concrete example program used this paper, a really interesting research topic could be to apply similar analysis techniques to real-world applications.
Maybe an analysis procedure similar to the one described in chapters \ref{sec:impl} and \ref{sec:ana} might work for the discovery of intended or unwanted (\textasciitilde bugs) cases of data leakage in publicly available software.

On a general level \sse may also be helpful for analyses of malware samples.
This would of course be a challenging task and require different project objectives and a much more advanced test setup, but could nonetheless be feasible considering \sse's holistic approach.


%Even more generally speaking
%, possibly even malware samples.
% towards more general suggestions, a really interesting 


\iffalse
§7	Outlook
		> Was man sonst noch Tolles machen könnte.
		> Anwendung auf echte Malware, etc…
\fi

%Other cool things one could do...

%Apply to real malware...
\section{Related Work}\label{sec:rel_work}

Banabic et al.~do bla... 
\cite{trojan14}


BitBlaze (see rel work in chip11s2e)

\iffalse
§8	Related Work
		> Was haben andere mit S2E in ähnlicher Richtung gemacht?
		> Z.B. "Finding Trojan Message Vulnerabilities in Distributed Systems”
\fi
\section{Conclusion}\label{sec:conclusion}




\iffalse
§9	Conclusion
		> Zusammenfassung: was habe ich gemacht?
		> Fazit: war die Sache sinnvoll und erfolgreich?
\fi

\iffalse %--------------------------- comment start!!!!!!!!!!!!!!!!!!!!!!!!!!!!!!!!
\section*{Section 1}

\lipsum[1-3] % Dummy text

\begin{align}
A =
\begin{bmatrix}
A_{11} & A_{21} \\
A_{21} & A_{22}
\end{bmatrix}
\end{align}

\lipsum[4] % Dummy text

%------------------------------------------------

\subsection*{Subsection 1}

\lipsum[7] % Dummy text

\begin{figure}
\begin{lstlisting}[language={[x86_64]Assembler}]
mov	al, 0x42
mov	rcx, 0x539
mov	rdi, 0x00c0ffeedeadbeef
doit:
xor	BYTE PTR [rdi+rcx], al
loop	doit
\end{lstlisting}
\caption{Simple xor-Loop}
\end{figure}

\lipsum[5] % Dummy text

\begin{itemize}
\item First item in a list
\item Second item in a list
\item Third item in a list
\end{itemize}

\lipsum[6] % Dummy text

%------------------------------------------------

\subsection*{Subsection 2}

\lipsum[7] % Dummy text

\begin{table}
\caption{Random table}
\centering
\begin{tabular}{llr}
\toprule
\multicolumn{2}{c}{Name} \\
\cmidrule(r){1-2}
First name & Last Name & Grade \\
\midrule
John & Doe & $7.5$ \\
Richard & Miles & $2$ \\
\bottomrule
\end{tabular}
\end{table}

%------------------------------------------------

\section*{Section 2}

\lipsum[4]

\begin{lstlisting}
#include <stdio.h>

int main(int argc, char **argv)
{
	puts("Hello World!");
	return 0;
}
\end{lstlisting}

\lipsum[8] % Dummy text

\begin{description}
\item[First] This is the first item
\item[Last] This is the last item
\end{description}

\lipsum[9] % Dummy text

\fi %--------------------------- comment end!!!!!!!!!!!!!!!!!!!!!!!!!!!!!!!!

%----------------------------------------------------------------------------------------
%	REFERENCE LIST
%----------------------------------------------------------------------------------------

\bibliographystyle{amsplain}
\bibliography{bib}

%----------------------------------------------------------------------------------------

\end{document}
