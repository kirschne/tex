\section{Outlook}\label{sec:outlook}

This chapter delineates a few suggestions for further interesting research topics related to the project of this paper, starting with concrete ideas for further analysis of the scenario defined in chapter \ref{sec:proj}.

\medskip
Due to the path explosion problems described above it would obviously be interesting to run \sse on a more powerful machine.
This would allow a more comprehensive observation strategy, like using more symbolic variables.

\medskip
Very fine-grained scripting of \sse's configuration file or even interfering in \sse's C++ source code could handle state forking and symbolic-concrete switching more intelligently.
Out of the box, \sse of course does not know our analysis goals and pursues every state.
Adapting this internal behaviour to personal needs should reduce execution time and effort significantly.

\medskip
During the analysis done for this paper, the server responding to \app's messages was available as binary and always running on the guest VM.
This is certainly a very unlikely situation when analysing real-world programs.
The server might not be available or respond with much latency.
It could potentially block the overload of requests triggered by our test execution, or we might not want to reveal our analysis efforts.

In all of these cases symbolic execution must become independent of the server's presence.
Besides, \app's execution tree currently does not depend on server responses.
If the application contained switches like `\textit{if srv\_response == x then do ...}', then the consistency model applied in chapters \ref{sec:impl} and \ref{sec:ana} would fail.

Such a scenario would require \sse execution to be configured for using a \textit{overapproximate consistency (RC-OC)} or \textit{control flow graph consistency (RC-CC)} model (see \cite{chip14s2e}, p.~26ff).
In the \app example, this would mean that responses from the server have to be intercepted and replaced with unconstrained symbolic variables.
Practically speaking, the whole function \textit{send\_data} would have to be replaced with a symbolic variable.
As a consequence, \sse would of course produce lots of states based on completely absurd potential server responses.
We simply do not know how the server responds to a certain client message nor how a valid server response looks like.
But this is the only way to make sure that \sse eventually also finds all valid server responses in the absence of any knowledge of how the server works. 


\medskip
Moving away from the the concrete example program used this paper, a really interesting research topic could be to apply similar analysis techniques to real-world applications.
Maybe an analysis procedure similar to the one described in chapters \ref{sec:impl} and \ref{sec:ana} might work for the discovery of intended or unwanted (\textasciitilde bugs) cases of data leakage in publicly available software.

On a general level \sse may also be helpful for analyses of malware samples.
This would of course be a challenging task and require different project objectives and a much more advanced test setup, but could nonetheless be feasible considering \sse's holistic approach.


%Even more generally speaking
%, possibly even malware samples.
% towards more general suggestions, a really interesting 


\iffalse
§7	Outlook
		> Was man sonst noch Tolles machen könnte.
		> Anwendung auf echte Malware, etc…
\fi

%Other cool things one could do...

%Apply to real malware...