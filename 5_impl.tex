\section{Implementation}\label{sec:impl}

As described in chapter \ref{sec:s2e}, working with \sse can be divided into a code selection and the actual analysis part.

In the \textbf{code selection phase} \sse is configured to focus on the program \app in user space only.
The whole environment will be treated as a black box.
% Possible calls into the kernel are assumed to work correctly.
Symbolic execution will only be applied inside the code of \app.

The next step is to precisely specify which variables shall be treated symbolically.
In general, as a start we want to make all user inputs symbolic.
For a closed source binary this can be done in several ways.
A command line tool can be called from a wrapper program which hands over symbolic arguments instead of concrete ones.
If user input is entered in a UI, a further code into the assembly code is necessary in order to find the memory locations where the corresponding variables are written.\todo{..}
%look for ... calls...
\sse can intercept execution of the specified memory addresses and replace them with symbolic values (state:writeMemorySymb(\ldots)).\todo{con?}
% consistency models?????

KLEE, the engine for symbolic execution, also requires some configuration.
As path search strategy this analysis uses a depth first search.
\todo{..}

\bigskip

Most logic in the \textbf{analysis part} relies on \sse's \textit{Annotations} plugin.
It allows fine monitoring and even fiddling with the execution state by annotating single instructions or functions in the program binary.



\iffalse
§5 	Implementation (of the test case using S2E)
		> Vorgehen
		> Verwendete Konsistenzmodelle
		> Arbeitsweise von Selektoren/Analysatoren
\fi