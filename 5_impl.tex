\section{Implementation}\label{sec:impl}

\todo{.}Klar machen, welche Plugins ich verwendet habe! Liste. Beschreiben, dass alles in Lua gecodet wird. Note: x86
General notes: before execution in \sse the sample application \app was highly optimised.
Blabla
\todo{.}

As described in chapter \ref{sec:s2e}, working with \sse can be divided into a code selection and the actual analysis part.

In the \textbf{code selection phase} \sse is configured to focus on the program \app in user space only.
The whole environment will be treated as a black box.
% Possible calls into the kernel are assumed to work correctly.
Symbolic execution will only be applied inside the code of \app.

The next step is to precisely specify which variables shall be treated symbolically.
In general, as a start we want to make all user inputs symbolic.
For a closed source binary this can be done in several ways.
A command line tool can be called from a wrapper program which hands over symbolic arguments instead of concrete ones.
If user input is entered in a UI, a further look into the assembly code is necessary in order to find the memory locations where the corresponding variables are written.\todo{..}
%look for ... calls...
\sse can intercept execution of the specified memory addresses and replace them with symbolic values (state:writeMemorySymb(\ldots)).\todo{con?}
% consistency models?????

KLEE, the engine for symbolic execution, also requires some configuration.
As path search strategy this analysis uses a depth first search.
\todo{..}

\bigskip
% -------------------------------------------------------------------
\begin{lstlisting}[language={[x86masm]Assembler}, escapeinside={(*§}{§*)}, basicstyle=\ttfamily\tiny, caption={Relevant parts of the function \textit{send\_data} in assembly code. Interesting calls are highlighted.}, label={lst:ass}, morekeywords={movl, movw, cmpl}]
0804940c (*§\hl{<\_Z9send\_dataPc>:}§*)
 		.
 		.
 8049448: e8 c3 fa ff ff       	call   8048f10 <socket@plt>
		.
		.
 80494e9: e8 12 f9 ff ff       	call   8048e00 (*§\hl{<connect@plt>}§*)
 80494ee: c1 e8 1f             	shr    $0x1f,%eax
 80494f1: 84 c0                	test   %al,%al
 80494f3: 74 0c                	je     8049501						<_Z9send_dataPc+0xf5>
 80494f5: c7 04 24 83 a0 04 08 	movl   $0x804a083,(%esp)
 80494fc: e8 f9 fe ff ff       	call   80493fa <_Z5errorPKc>
 8049501: 8d 85 d8 fe ff ff    	lea    -0x128(%ebp),%eax
 8049507: 89 04 24             	mov    %eax,(%esp)
 804950a: e8 21 fa ff ff       	call   8048f30 <strlen@plt>
 804950f: 89 44 24 08          	mov    %eax,0x8(%esp)
 8049513: 8d 85 d8 fe ff ff    	lea    -0x128(%ebp),%eax
 8049519: 89 44 24 04          	mov    %eax,0x4(%esp)
 804951d: 8b 45 f0             	mov    -0x10(%ebp),%eax
 8049520: 89 04 24             	mov    %eax,(%esp)
 8049523: e8 58 f9 ff ff       	call   8048e80 (*§\hl{<write@plt>}§*)
		.
		.
 804955b: 8d 85 d8 fe ff ff    	lea    -0x128(%ebp),%eax
 8049561: 89 44 24 04          	mov    %eax,0x4(%esp)
 8049565: 8b 45 f0             	mov    -0x10(%ebp),%eax
 8049568: 89 04 24             	mov    %eax,(%esp)
 804956b: e8 70 f9 ff ff       	call   8048ee0 (*§\hl{<read@plt>}§*)
		.
		.
\end{lstlisting}
\bigskip

Most logic in the \textbf{analysis part} relies on \sse's \textit{Annotations} plugin.
It allows fine monitoring and even fiddling with the execution state by annotating single instructions or functions in the program binary.

For the scenario in this paper all instructions which handle connections to the internet are of particular interest.
The C library call \textit{connect} helps to identify where the program is connecting to, and the corresponding \textit{write} and \textit{read} calls allow to find out what data is sent and received in this connection.
Memory addresses of these relevant calls can be retrieved from the assembly code.
Since all calls concerned with connecting to the internet seem to take place in a single function named \textit{send\_data}, calls to this function as a whole shall be monitored, too.
Listing \ref{lst:ass} shows relevant parts of the function \textit{send\_data} in assembly code.
%, together with comments...

Verbindung\todo{?}

\bigskip
% -------------------------------------------------------------------
\begin{lstlisting}[language={[5.0]Lua}, basicstyle=\ttfamily\footnotesize, caption={Configuration of the \textit{Annotations} plugin (part). Defines the instructions to be monitored and actions to trigger upon execution of these instructions. Note the link to the binary in listing \ref{lst:ass} via memory addresses.}, label={lst:ann_def}]
pluginsConfig.Annotation = {
  annotation_write = {
    active=true,
    module="prog",
    address=0x8049523, -- write()
    instructionAnnotation="do_write",
    beforeInstruction=true,
    switchInstructionToSymbolic=false
  },
  annotation_read = {
    active=true,
    module="prog",
    address=0x804956b, -- read()
    instructionAnnotation="do_read",
    beforeInstruction=false, 
    switchInstructionToSymbolic=false
  },
  annotation_send_data = {
    active=true,
    module="prog",
    address=0x804940c, -- send_data()
    callAnnotation="call_ann",
    paramcount = 1
  }
}
\end{lstlisting}
\bigskip


After providing \sse's \textit{Annotations} plugin with all relevant memory addresses plus a little more configuration (shown in listing \ref{lst:ann_def}), we can now execute code every time the program runs into one of the interesting instructions.
At this point, we will \ldots

1.) log that instruction $X$ was reached.
Together with other \sse output this information shows which execution paths run into the \textit{send\_data} method and when they do so.

2.) save information about the executed instruction in the current plugin state.
This allows to check whether there have been prior connections to the internet when the program runs into the next annotated instruction.

3.) read out parameters of the called function.
The function \textit{send\_data}, for instance, is called with one parameter, which appears to be a memory location.
Now \sse's analysis interfaces can be employed to dynamically find out what is written in the respective memory.

4.) read (or write) registers ($EAX$, $EBP$, ...).
Thus \sse users can check and even manipulate the current execution state, similar to debug situations.
%... Check out call conventions of write, ...
%This may either be used to check the current internal state, or...

5.) switch variables (memory locations) from concrete to symbolic mode (or vice versa).
... Nice for controlling consistency models. Not used currently.\todo{.}

\bigskip

Listing \ref{lst:send_data}, for instance, shows the Lua function which is triggered on every execution of the assembly function \textit{send\_data} (see listing \ref{lst:ann_def} for the registration of this function and its coupling to a specific memory address).
Because \textit{send\_data} was registered as a function call in listing \ref{lst:ann_def}, \sse's interfaces facilitate the extraction of all parameters via \textit{state:readParameter()}.
A sample test run shows that $arg0$ is clearly a memory address, so the next step must be to examine what data is stored at this location.
The method \textit{print\_mem} does just that and additionally brings the contents into a human-readable format.
Then the Lua script stores information that this function was called in the current execution state.
Doing so, the next call will notice that the method has been executed before, increment the counter and write this fact into a log file.

The $else$ branch will be called when the function returns.
Following common calling conventions, we expect return values to be transmitted via the register $EAX$ (and $EDX$ if bigger than four bytes), hence $EAX$ could also hold interesting information.
Unfortunately, this does not seem to be the case as $EAX$ is always $0$ except in some error cases.
So \textit{send\_data} is probably just returning a status integer value.







\bigskip
% -------------------------------------------------------------------
\begin{lstlisting}[language={[5.0]Lua}, basicstyle=\ttfamily\footnotesize, caption={Lua function executed upon every call of the function send\_data(). See lines 18 - 24 in listing \ref{lst:ann_def} for the registration of \textit{call\_ann}.}, label={lst:send_data}]
function call_ann (state, curPlgState)
  if curPlgState:isCall() then
    arg0 = state:readParameter(0);
    print ("Calling function with with arg0=" .. ("%x"):format(arg0));
    -- print_mem dereferences the pointer arg0 and converts the memory behind that address into a string.
    print_mem(state, curPlgState, arg0);
    -- saving info about the no. of connections in this state so far
    local no =curPlgState:getValue("no");
    if no == nil then
      no = 0
    end
    no = no + 1;
    curPlgState:setValue("no", no);
    print ("\n\tNo. of connections on this path so far: " .. no);
  else
    -- Called when the function send_data returns. The return value can be read from the EAX register.
    local eax =state:readRegister("eax");
    print("EAX: " .. ("%x"):format(eax));
  end
end
\end{lstlisting}
\bigskip

%Figure \ref{} shows relevant parts of the assembly code. Figure \ref{} presents excerpts from the resulting analysis plugin.

\bigskip

In addition to the \textit{Annotations} plugin, analysis in this paper also employs the \textit{TestCaseGenerator}.
For each execution path, it finds concrete inputs for all symbolic variables.
Those will be printed upon termination of the path and, apart from helping to understand the program, may serve as input for further testing.

\bigskip

Naturally, an important step for understanding symbolic execution of \app is to log and later interpret output of KLEE, the symbolic execution engine used in \sse.
The behaviour of KLEE can also be controlled via \sse's Lua configuration file.



\iffalse
§5 	Implementation (of the test case using S2E)
		> Vorgehen
		> Verwendete Konsistenzmodelle
		> Arbeitsweise von Selektoren/Analysatoren
\fi