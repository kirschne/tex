\section{Related Work}\label{sec:rel_work}

One interesting application of \sse is the work of Banabic, Candea and Guerraoui, who use selective symbolic execution for finding Trojan message vulnerabilities in distributed systems \cite{trojan14}.
In a server-client environment they define `Trojan messages' as messages which are accepted as valid by the client but cannot be generated by a correctly working server.
This behaviour may arise from bugs in the client, for instance a forgotten check of an input field.
Such a case may be a vulnerability in the distributed system.

Banabic et al.~use \sse to explore execution paths of both the server and the client.
Thus they find all messages that can be generated by correct servers as well as those that are accepted by correct clients.
If these two sets of messages do not coincide exactly, a Trojan message vulnerability has been found.

\medskip
Chipounov and Candea also published papers of how to use \sse to reverse engineer binary device drivers \cite{chipounov2010reverse, chipounov2011enabling}.


\medskip
Symbolic execution is employed by many other research groups, too.
Caballero et al., for instance, make use of symbolic execution exploration techniques for finding bugs in malware \cite{caballero2010input}.
Most interesting in their work is how symbolic execution can help to analyse malware functionality which is hard to understand, like encryption, checksums, obfuscated code, etc.


\medskip
Caballero's work bases on BitBlaze, an analysis platform for finding security problems in binary programs.
Similar to \sse, it uses virtualisation and symbolic execution and offers some of \sse's consistency models.
BitBlaze was introduced by Song et al.~in \cite{song2008bitblaze}.

\iffalse
§8	Related Work
		> Was haben andere mit S2E in ähnlicher Richtung gemacht?
		> Z.B. "Finding Trojan Message Vulnerabilities in Distributed Systems”
\fi