\section{Project Idea and Research Questions}\label{sec:proj}

The practical part of this project strives to explore privacy issues in a sample binary.

In order to make life easier, many people use little freeware applications on a regular basis.
But most of these programs are proprietary and have to be trusted without any knowledge of their functioning.
Real malware (Trojan horses, spyware, ...) is usually detected rather quickly by anti-virus software and can often be blocked effectively.
However, between unambiguous malware and thoroughly benign software many shades of grey can be found.

This work will focus on the scenario that an application (intentionally or unintentionally) leaks delicate private data without the user's consent or knowledge.


Due to the difficulty of finding a real-world program which shows exactly this desired behaviour and also in general the complexity of real-world applications, the showcase described here bases on a little self-written program.

The scenario works as follows: The freeware tool \app found in the internet claims to be handy for estimating your personal tax load.
The tool announces to display a decent advertisement on every startup, which suggests a common and reasonably sound business model.
It promises to treat all personal data confidentially, but since the application is closed source, of course this claim cannot be verified easily.

To make sure that \app does not compromise the user's privacy, it shall be analysed using selective symbolic execution techniques and the \sse platform.

\bigskip

For a general overview, often a good first step is to have a glance at the program's assembly code.
Since static code analysis of large programs is a very time-consuming task\todo{obf}, that is not what we want to do here.
Nevertheless, some useful information can be retrieved from the assembly code quite easily.
A look at the list of invoked external libraries shows that \app communicates over network sockets, i.e.~uses the C library functions connect, write, etc.\todo{cmd}
That is of course not really surprising as the program relies on advertisements as a business model.
However, such connections to the internet are always delicate and could potentially leak personal data.
Therefore further analysis in this paper will focus on the following concrete questions:
\begin{enumerate}
  \item When (under which conditions) are connections to the internet established?
  \item What data is transferred in these connections?
  \item Does a connection to the internet leak personal data?
\end{enumerate}
These questions could of course be tackled via many different analysis techniques.
Simply debugging the program should already lead to a rough understanding of the program's behaviour.
Applying techniques like fuzzing \todo{cit} could increase code coverage with the goal of not missing exotic or covert execution paths.
This paper, however, will rely on analysis using selective symbolic execution as described in chapter \ref{sec:s2e}.\todo{why}

\bigskip

Before proceeding with the actual analysis, a few notes on the general setup: 
Since \sse works best with x86 guest architectures, \app was compiled on a x86 Debian system using standard GNU tools (g++).
For the sake of convenience in this sample scenario the resulting application binary is neither overly optimised nor in any way obfuscated.
As a consequence, the assembly code still contains meaningful variable and function names.


\iffalse
§4	Project idea: explore privacy issues in a sample binary
		> Plan darlegen: Programme könnten unerwünscht Infos preisgeben.
		> Daher: Eigenes kleines Programm, das … macht.
\fi