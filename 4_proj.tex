\section{Project Idea}\label{sec:proj}

The practical part of this project strives to explore privacy issues in a sample binary.

In order to make life easier, many people use little freeware applications on a regular basis.
But most of these programs are proprietary and have to be trusted without any knowledge of their functioning.
Real malware (Trojan horses, spyware, ...) is usually detected rather quickly by anti-virus software and can often be blocked effectively.
However, between unambiguous malware and thoroughly benign software many shades of grey can be found.

This work will focus on the scenario that an application (intentionally or unintentionally) leaks delicate private data without the user's consent or knowledge.


Due to the difficulty to find a real-world program which shows exactly this desired behaviour and also in general the complexity of real-world applications, the showcase described here bases on a little self-written program.

The software works as follows:

What we do not want is...


All analysis will be done using the \sse platform.
\todo{m?}

\iffalse
§4	Project idea: explore privacy issues in a sample binary
		> Plan darlegen: Programme könnten unerwünscht Infos preisgeben.
		> Daher: Eigenes kleines Programm, das … macht.
\fi