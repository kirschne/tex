\bigskip

\section{Introduction}

% use cases (sel symb exe paper kap 2) hier hinein?

Frequently developers need to understand software systems.
In a very simple case they just analyse their own code or test the interaction of own programs with other components or with the surrounding environment in general.
Testing self-written programs conceptually permits the application of the whole arsenal of analysis techniques.


Things become interesting when analysis has to be performed without access to source code or documentation.
Scenarios for this situation include the need to check proprietary third party software for interoperability on existing servers, performance, unwanted side effects, and much more.
Security-critical environments additionally require reliable guarantees of the benignity of all employed software.


One mighty solution for such system analysis is the \sse platform developed at the Swiss Federal Institute of Technology in Lausanne (EPFL) \cite{chip11s2e}.
Its goal is to provide a tool set for rapid development of analysis tools like performance profilers, bug finders, reverse engineering solutions and the like \cite{chip12s2e}.
\sse combines several key characteristics:

1.) The ability to explore entire \textit{families of execution paths} helps to obtain reliable information about the target system.
Abstracting from single-path exploration to sets of execution paths which share specific properties is vital for predictive analyses.
This technique can for example prove the non-existence of critical corner cases which might be overlooked by other testing strategies.
% ...and hence provides more reliable results.
%Particularly for predictive tasks the analysis of sets of execution paths which share specific properies
%This is often necessary for predictive analyses, where classical single-path exploration techniques fail to provide reliable results.

2.) \textit{In-vivo analysis}, meaning the analysis of a program within its real-world environment (libraries, kernel, drivers, etc.), facilitates extremely realistic and accurate results.

3.) Working directly on \textit{binaries} further increases the degree of realism in system analyses, as it allows to include closed source modules into the investigation.

\bigskip

As an exemplary showcase for the power of the \sse platform and its underlying concepts this paper will perform a thorough analysis of a binary file in user mode.
The scenario assumes that this program was found somewhere in the internet and claims to be a useful freeware tool.
\sse shall help to investigate whether the binary compromises the user's privacy, for instance by leaking private data to the internet.


%What I do is bla...
%...justify the choice of \sse as platform for system analysis.\todo{?}

\bigskip

Chapter \ref{sec:s2e} explains the theoretical concepts of selective symbolic execution.
Chapter \ref{sec:platform} then introduces \sse, the platform which builds upon all techniques described before.
Coming to the practical part, chapter \ref{sec:proj} lays out the concrete analysis scenario and formulates research questions.
Chapter \ref{sec:impl} describes how \sse can be applied in this scenario, followed by an explanation of \sse results in chapter \ref{sec:ana}.
Possible further research related to this topic is mentioned in chapter \ref{sec:outlook}, together with some selected related work in chapter \ref{sec:rel_work}.
Finally, chapter \ref{sec:conclusion} summarises and concludes this paper.

%\newpage
\iffalse
§1	Introduction (Motivation)
		> Motivation, warum es einen Bedarf für etwas wie S2E gibt
		> Kurze Vorschau, was man damit Tolles machen kann
			-> Pfadanalyse -> Prediction of system behaviour
			-> Analyse in natürlicher Umgebung -> “in-vivo”

§2	Selective Symbolic Execution
		> Theorie-Teil
		> Was ist Symbolic Execution?
		> Was kann Selective Symbolic Execution besser?
			(Concrete -> symbolic transition usw.)
		> Konsistenzmodelle (wird hier evtl. schwierig, das richtige Maß 
			zu finden, um die Sache auf wenig Platz zu verstehen)

§3	The S2E Platform
		> Architektur
		> Funktionsweise
		> Selektoren + Analysatoren

§4	Project idea: explore privacy issues in a sample binary
		> Plan darlegen: Programme könnten unerwünscht Infos preisgeben.
		> Daher: Eigenes kleines Programm, das … macht.

§5 	Implementation (of the test case using S2E)
		> Vorgehen
		> Verwendete Konsistenzmodelle
		> Arbeitsweise von Selektoren/Analysatoren

§6	Interpretation of S2E analysis output
		> Execution Traces
		> Gefundene Privacy-Probleme
		> Eventuell nicht gefundene Sachen
		> Probleme bei der Analyse

§7	Outlook
		> Was man sonst noch Tolles machen könnte.
		> Anwendung auf echte Malware, etc…

		> Kommt das hier zu früh? Evtl. auch in den Schlussteil...

§8	Related Work
		> Was haben andere mit S2E in ähnlicher Richtung gemacht?
		> Z.B. "Finding Trojan Message Vulnerabilities in Distributed Systems”

§9	Conclusion
		> Zusammenfassung: was habe ich gemacht?
		> Fazit: war die Sache sinnvoll und erfolgreich?
\fi






