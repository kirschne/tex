\bigskip
\bigskip

%\newpage
\iffalse
§1	Introduction (Motivation)
		> Motivation, warum es einen Bedarf für etwas wie S2E gibt
		> Kurze Vorschau, was man damit Tolles machen kann
			-> Pfadanalyse -> Prediction of system behaviour
			-> Analyse in natürlicher Umgebung -> “in-vivo”

§2	Selective Symbolic Execution
		> Theorie-Teil
		> Was ist Symbolic Execution?
		> Was kann Selective Symbolic Execution besser?
			(Concrete -> symbolic transition usw.)
		> Konsistenzmodelle (wird hier evtl. schwierig, das richtige Maß 
			zu finden, um die Sache auf wenig Platz zu verstehen)

§3	The S2E Platform
		> Architektur
		> Funktionsweise
		> Selektoren + Analysatoren

§4	Project idea: explore privacy issues in a sample binary
		> Plan darlegen: Programme könnten unerwünscht Infos preisgeben.
		> Daher: Eigenes kleines Programm, das … macht.

§5 	Implementation (of the test case using S2E)
		> Vorgehen
		> Verwendete Konsistenzmodelle
		> Arbeitsweise von Selektoren/Analysatoren

§6	Interpretation of S2E analysis output
		> Execution Traces
		> Gefundene Privacy-Probleme
		> Eventuell nicht gefundene Sachen
		> Probleme bei der Analyse

§7	Outlook
		> Was man sonst noch Tolles machen könnte.
		> Anwendung auf echte Malware, etc…

		> Kommt das hier zu früh? Evtl. auch in den Schlussteil...

§8	Related Work
		> Was haben andere mit S2E in ähnlicher Richtung gemacht?
		> Z.B. "Finding Trojan Message Vulnerabilities in Distributed Systems”

§9	Conclusion
		> Zusammenfassung: was habe ich gemacht?
		> Fazit: war die Sache sinnvoll und erfolgreich?
\fi

\section{Introduction}




