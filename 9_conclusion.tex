\section{Conclusion}\label{sec:conclusion}

This paper described an approach for finding data leaks in a binary file using selective symbolic execution techniques provided by the \sse platform.
The investigated binary file, a self-written application called \app, claimed to be a useful, ad-supported freeware tool, but under certain circumstances turned out to transmit all confidential user data to a server in the internet.

\sse proved to be a mighty platform for universal testing of binary programs and even entire systems.
With a little practice, its code selection and analysis interfaces can be controlled conveniently via Lua configuration scripts.
After adequate configuration, \sse's standard analysis output already gives solid insights into the functioning of the investigated binary. Analysis output can be enhanced by writing custom Lua code.

But even under the laboratory settings employed here, with a very small and simple binary, path explosion of the symbolic execution turned out to be a big problem concerning time and the resources it takes to explore all execution states.
Investigation of real-world applications would require a more intelligent handling of \sse's path forking behaviour, with the goal of pursuing only states which are relevant for the analysis.



\iffalse
§9	Conclusion
		> Zusammenfassung: was habe ich gemacht?
		> Fazit: war die Sache sinnvoll und erfolgreich?
\fi